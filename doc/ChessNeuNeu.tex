\documentclass[a4paper,10pt]{article}
\usepackage[left=2cm,right=2cm,top=2cm,bottom=2cm]{geometry}
\pagestyle{empty}\parindent=0pt
\usepackage{chessboard, xskak}
\usepackage{amsmath}
\usepackage{hyperref}
\usepackage[ruled,longend]{algorithm2e}
\begin{document}

%
\newcommand{\norm}[1]{\left\lVert#1\right\rVert}
\SetKwInOut{Init}{initialisation}

\section*{Introduction}

The goal of this document is to formalize reflections and explain algorithms
developed in the code source of this
\href{https://github.com/Lecrapouille/ChessNeuNeu}{GitHub project}. The goal of
this project is to check if it is possible for a computer to learn chess rules,
or in other words, if it is possible to make it played legal piece moves
\textbf{without cheating}. The goal of this project is not to make predict moves
or learn chess strategy to a computer (chessboard evaluation) because this has
already been \href{https://github.com/ashudeep/ConvChess}{made}. It is also
interesting to compare how learning is expansive (in term of energy, iterations)
in comparison to a human.

\section{Notation of a chess movement}
A chessboard is a $8 \times 8$ grid and therefore has 64 squares. For this
document we define a chess movement has a pair of squares inside the board:
origin and destination. We will note it as $(o, d)$ where $o$ and $d$ are
positive integers and $< 64$. For example the green move of the rook on the
next picture is \wmove{Re4-g4} with e4 is the 32\textsuperscript{th} square
(starting from the 0\textsuperscript{th} square: a8) and g4 is the
34\textsuperscript{th} square. Therefore $(o, d) = (32, 34)$.

\section{Teach moving a single piece on an empty board}

Learning to move a single piece on an empty chess board with machine learning is
the most simple case for teaching chess rules to a computer. We use a fully
connected neural network architecture.

We place randomly the desired piece on the empty chessboard. The computer tries
a random move (legal or illegal) and a supervisor validates or invalidates the
move which make reinforcing synapses: weights of synapses of the network are
decreased or maintained. The supervisor knows the chess rules (while for this
case slightly modified for accepting the non-presence of kings on the board).

\chessboard[
  setpieces={Re4},
  pgfstyle=straightmove,
  arrow=stealth,
  linewidth=.25ex,
  padding=1ex,
  pgfstyle=straightmove,
  shortenstart=1ex,
  showmover=true,
  %
  color=red!75!white,
  markmoves={e4-g5},
  %
  color=green!75!white,
  markmoves={e4-g4}
]

The picture shows examples of random moves for a white rook: -- in red the
illegal move refused by the supervisor; -- in green the accepted move.

\subsection{Algorithm}

Let $A$ the matrix holding weights of the synapses, let $e$ the input of the
neural network and $q$ its output. $A$ holds all combination of movements
therefore $A$ is a $64 \times 64$ matrix. $e$ is a vector of $64$ elements
(squares of the chessboard): we place the value $1$ to the square holding the
piece (therefore the origin $o$ of the movement) and $0$ when empty. $q$ is the
output: it will contain the normalized probabilities of the destination squares
for finishing the movement.

\newpage
The algorithm \ref{alg:TrainSynapses} for training a piece to move is the
follow.

\begin{algorithm}
  \label{alg:TrainSynapses}
  \DontPrintSemicolon
  $A_{i,j} = 1 \text{ where } i,j \in \{0 \dotsc 63\}$\;
  \For{$X$ \text{iterations}}
  {
    $o \gets \text{random a square}$\;
    $d \gets \text{SynapsesPlay(}A,o\text{)}$\;
    $A_{o,d} =
    \begin{cases}
      A_{o,d} + 1,& \text{if } (o, d) \text{\,is a legal move} \\
      \max(1,A_{o,d}) - 1,& \text{else } (o, d) \text{\,is an illegal move}
    \end{cases}
    $\;
  }
  \caption{Train Synapses}
\end{algorithm}

All synapses $A_{i,j}$ are set with a weight of 1. We random the origin $o$ of
the movement. From $o$ and $A$ we random for the destination of the move.  We
decimate synapses when the obtained move is an illegal chess move.  We reinforce
synapses when a legal move is found. The sub algorithm \ref{alg:SynapsesPlay}
gets the destination of the move.

\begin{algorithm}
  \label{alg:SynapsesPlay}
  \DontPrintSemicolon
  \KwIn{
    $A$: synapses, $o$: origin of the move\;
  }
  \KwOut{
    $d$: destination of the move\;
  }
  Initialization:\;
  $e_i =
  \begin{cases}
    1,& \text{if } i = o \\
    0,& \text{else}
  \end{cases} \text{ where } i \in \{0 \dotsc 63\}$\;
  Matrix production with normalization of the result:\;
  $q = A \times e$\;
  $S = \sum_{i=0}^{63}q_i$\;
  $q_i \gets q_i / S, \forall i \in \{0 \dotsc 63\}$\;
  Randomize the destination move:\;
  $p \gets \text{random(0,1)}$\;
  $q = 0$\;
  \For{$(d=0$; $d<64$; $d=d+1)$}
  {
    $q = q + q_d$\;
    \If{$q \geq p$}
       {
         \Return{$d$}
       }
  }
  \Return{no move}
  \caption{Synapses Play}
\end{algorithm}

Knowing $o$, we initialise the input vector $e$ by placing $1$ to the
$o$\textsuperscript{th} element and 0 others. We compute probabilities $q$ for
the destination of the move by doing the matrix product of $A$ and $e$. From
normalized $q$, we random a the destination of the move by randomizing a number
between 0 and 1.

The following figure shows an extract of the matrix $A$ for a Rook after $10^5$
iterations.

\begin{verbatim}
      A8    B8    C8    D8    E8    F8    G8    H8    A7    B7    C7 ... H1
A8     0     8    60   128    69    47   100   313    77     1     1
B8     6     0     2   263    75   111   138   156     0    75     0
C8    54     3     0   111   139    15    48    96     1     0   183
D8   132   254   107     0   124     9   236    21     0     1     1
E8    73    78   156   130     0    88    28    17     0     0     0
F8    56   115    18    10    89     0    68   142     0     1     0
G8   105   150    52   244    35    81     0    21     0     1     0
...
H1
\end{verbatim}

We read this matrix as follow: the column is the origin of the move and the row
the destination. Let suppose a Rook placed on the a8 square. Illegal moves like
\wmove{Ra8-a8} is correctly understood: the weight is $0$. Legal moves such as
\wmove{Ra8-b8} or \wmove{Ra8-c8} are correctly understood: weights are
$>0$. Nevertheless, we can see that the matrix still have initial values $1$
(for example \wmove{Ra8-b7}) meaning that this branch has not been explored and
potentially could produce an illegal move.

\subsection{Code source}

The code of the supervisor is made in src/Chess/Rules.cpp and the code of the
training is made in src/Players/NeuNeu.cpp. Enable the macro RANDOM\_MOVES.

\subsection{Optimized algorithm}

With this simple algorithm that we need to iterate over millions of iterations
to obtained a correct weights for synapses $A$ and explore all cases. If not, a
branch of movements may not have been tried and not decimated. As consequence
the computer may choose an illegal movement. This is of course not acceptable
because considered as cheating by a human player.

We could improve the algorithm of the gradient descent but a simpler idea comes
immediately: why not simply iterate on all $64 \times 64$ cases and eliminate
definitively synapses when the move is illegal ? This solution works nicely and
only takes 4096 iterations. Here synapses for the Rook:

\begin{verbatim}
    A8  B8  C8  D8  E8  F8  G8  H8  A7  B7  C7 ... H1
A8   0   1   1   1   1   1   1   1   1   0   0 ...
B8   1   0   1   1   1   1   1   1   0   1   0
C8   1   1   0   1   1   1   1   1   0   0   1
D8   1   1   1   0   1   1   1   1   0   0   0
E8   1   1   1   1   0   1   1   1   0   0   0
F8   1   1   1   1   1   0   1   1   0   0   0
G8   1   1   1   1   1   1   0   1   0   0   0
...
H1
\end{verbatim}

\subsection{Real game}

This algorithm can be applied for all type of pieces: Rook, Knight, Bishop,
Queen and King. Training a single color is sufficient for playing the
opponent. For Pawns we have to create a network by type of color because
directions are opposite. Note that the color of the square for Bishops is
agnostic.

Once trained, the machine can almost play against a human. We random a piece
along the available pieces. This makes the origin of the move. We apply the
algorithm \ref{alg:SynapsesPlay} SynapsesPlay for the destination. Because we
did not teach not to move a piece to move to a piece of the same color, the
supervisor, knowing chess rules, makes random a new move.

\subsection{Code source}

The code of the supervisor is made in src/Chess/Rules.cpp and the code of the
training is made in src/Players/NeuNeu.cpp. Disable the macro RANDOM\_MOVES.

\section{Teach to move a single piece on an obstructed board}

The next step is to teach how to move a piece with obstacles. By obstacle we
mean the presence of another pieces of the same color. Again, to start with the
simplest case, we focus with a $8 \times 1$ chessboard (therefore to a
1-Dimension problem) and we suppose that the moving piece can only go to the
right. Chess rules concerning obstacles is maintained. The two next pictures
show the principle.

\chessboard[
    maxfield=h1,
    setpieces={Rb1,Pe1,Pg1},
%
    pgfstyle=straightmove,
    arrow=stealth,
    linewidth=.25ex,
    padding=1ex,
    color=green!75!white,
    pgfstyle=straightmove,
    shortenstart=1ex,
    showmover=true,
    markmoves={b1-d1}
]

On this picture, the green arrow shows a valid move. The a1 square is not
explored for this problem.

\chessboard[
    maxfield=h1,
    setpieces={Rb1,Pe1,Pg1},
%
    pgfstyle=straightmove,
    arrow=stealth,
    linewidth=.25ex,
    padding=1ex,
    color=red!75!white,
    pgfstyle=straightmove,
    shortenstart=1ex,
    showmover=true,
    markmoves={b1-f1}
]

On this picture, the red arrow shows an invalid move because in chess rule a
Rook, contrary to the Knight, cannot jump over other pieces.

\subsection{Convolutional Neural Network}

Contrary to the previous section where a fully connected neural network have
been used, we are going to use a convolutional neural network (CNN). CNN have
done a great improvement for machine learning concerning image recognition: they
allow to create numerous small kernel filters detecting small portion of images
(such as edges). These filters have the property to be position invariant inside
the image.

CCN are so powerful that nowadays many non grid machine learning problems are
translated into a grid problem. Fortunately for us, a chessboard is already a
2-Dimension grid.

The following
\href{http://brohrer.github.io/how_convolutional_neural_networks_work.html}{document}
is a very well good introduction to CNN and their convolution and pooling
functions.

Our CNN is made of the following layers:
\begin{itemize}
\item[$\bullet$] Input: two layers representing two $8 \times 1$
  chessboards. The first chessboard/layer contains the blocking figures. We do
  not care of the type of the piece (Pawns, ...). Only their presence on squares
  is important: $1$ and their absence is $0$ for each square. The second
  chessboard/layer contains the moving piece.  Layer is in fact we increase
  tensor of 7 elements (passing from 8 to 15) because of the next layer: conv.
\item[$\bullet$] First layer: Conv. Padding is 7 because the chessboard is not
  infinite and we have to take care of borders.
\item[$\bullet$] Second layer: max pool.
\item[$\bullet$] Thrird layer: fully connected layer.
\item[$\bullet$] Output: returns the probability of the number of squares for the
  movement from 0 (meaning a move of 0 square. Or in the case the piece cannot
  move) to
\end{itemize}

\subsection{Training}

We do not a database of chess games for training this CNN. The problem is so
simple that we can generate chessboards and their solution for the supervisor.

x = zeros(8,1,2,1)
x[:,1,1,1] = Int.(floor.(0.5 .+ rand(8)))

p = Int.(floor.(1 + 8 * rand(1)[1]))
x[p,1,2,1] = 1

The supervisor calcy training function will return y

For example in the previous figures y will be $[0 0 1 0 0 0 0]$


Say differently, this CNN learns the distance to the first blocking piece.

\subsection{Testing}



\subsection{Newbie error for the training}

For this problem it is important to generate a board with several blocking
pieces. If we generate only a single blocking piece, the CNN will learn the
distance between the moving piece to the blocking piece but not the distance
between the moving piece to the \textbf{first} blocking piece. As consequence if
training with a single blocking piece and testing on a obstructed board, the CNN
will not be able to detect the first blocking piece and will not be able to move
correctly the piece.

While one blocking piece the CNN will converge faster and explore more
possibilities than an obstructed board, this is a different problem.

\subsection{Drawback of this method}

The probability of having a chessboard with 1 or 0 blocking pieces is very few
contrary to a chessboard where the piece cannot move. As consequence, we have to
be sure to generate a sufficient number of training where this case is produced
for the training. For example with 20 thousands of iterations we can obtain 40
cases of empty chessboard. As consequence a bad choice of iterations can produce
unexplored case. A good choice of data for training shall include edges cases.

\subsection{Test human} convolutional

What we succeeded to make understand to the machine is the algorithm of the
distance to first blocking piece placed on the right. This simple algorithm
needed around 5 lines of Julia or C.

We can compare in how many iterations are needed for a human to understand the

It can be interesting to test humans with the same. Of course, the problem has
to be shown differently. Firstly, human does not have to see it is a chessboard.
So replace the chessboard of figure XX by a simple grid uniformly white and
replace the piece to be moved (Rook) by a circle and blocking pieces (Pawns) by
a cross. Show this figure to a human and simply ask to him ``give me a number
between 0 and 7''. Repeat again until he understand that the answer is the
number of empty squares to the first blocking pieces. Count the
number of iterations needed.

A human will not need thousand of iterations before understanding the solution
because a human brain is good for localizing lines in a pattern. A more
difficult variant of this problem could be the distance to the second blocking
piece. The human brain is less good for this job.

\subsection{Code source}

The code of this section has been made with Julia (version >= 1.0) and the
Knet package. See ChessNeuNeu/scripts/ChessKnet.jl

\end{document}
